\documentclass[10pt,a4paper]{article}
\usepackage[utf8]{inputenc}
\usepackage[a4paper,%
            left=1in,right=1in,top=1in,bottom=1in]{geometry}
\setlength{\headsep}{0.25in}

\usepackage{amsthm}

\usepackage{graphicx}
            
\usepackage[english]{babel}

\theoremstyle{theorem}
\newtheorem{theorem}{Theorem}
\newtheorem{lemma}{Lemma}
\newtheorem{corollary}{Corollary}
\newtheorem{case}{Case}

\theoremstyle{definition}
\newtheorem{definition}{Definition}

\usepackage{mathtools}
\DeclarePairedDelimiter\bra{\langle}{\rvert}
\DeclarePairedDelimiter\ket{\lvert}{\rangle}
\DeclarePairedDelimiterX\braket[2]{\langle}{\rangle}{#1 \delimsize\vert #2}

\usepackage{amsmath}
\usepackage{amsfonts}
\usepackage{amssymb}
\usepackage{fancyhdr}

\DeclareMathOperator{\interior}{int}

\newcommand{\Tau}{\mathcal{T}}

\newenvironment{amatrix}[1]{%
  \left(\begin{array}{@{}*{#1}{c}|c@{}}
}{%
  \end{array}\right)
}

\usepackage{calligra}
\DeclareMathAlphabet{\mathcalligra}{T1}{calligra}{m}{n}
\DeclareFontShape{T1}{calligra}{m}{n}{<->s*[2.2]callig15}{}

\newcommand{\scripty}[1]{\ensuremath{\mathcalligra{#1}}}

\pagestyle{fancy}
\author{Jeremiah Givens}
\newcommand{\subject}{Metric Spaces}
\newcommand{\Date}{9/2/2021} 
\makeatletter
\rhead{{\small\@author}}
\lhead{{\small\subject}}
\chead{{\large In Class Exercises 8/31/2021}}
\cfoot{}
\rfoot{\thepage}
\lfoot{\today}

\renewcommand{\theequation}{\arabic{equation}}

\begin{document}
\begin{theorem}
Let $(X, \Tau)$ be a topological space, and let $A,B \subseteq X$. Then 
\begin{enumerate}
\item $A \subseteq B \implies \mathring{A} \subseteq \mathring{B}$
\item $\mathring{A} \cap \mathring{B} \subseteq \interior(A \cup B)$
\item $\interior(A \cap B) \subseteq \mathring{A} \cap \mathring{B}$
\item $\mathring{A} = \bigcup \{U : U \in \Tau \land U \subseteq A \}$.
\end{enumerate}
\end{theorem}

\begin{proof}
\begin{enumerate}
\item Suppose $A \subseteq B$.  Then
\begin{align*}
p \in \mathring{A} &\iff (\exists U(p) \in \Tau)(U(p) \subseteq A)\\
&\implies(\exists U(p) \in \Tau)(U(p) \subseteq B) && \text{Since } A \subseteq B\\
&\iff p \in \mathring{B} \text{,}
\end{align*}
as desired.
\item We have 
\begin{align*}
p \in \mathring{A} \cap \mathring{B} &\iff p \in \mathring{A} \land p \in \mathring{B}\\
&\iff (\exists U_1(p) \in \Tau)(U_1(p) \subseteq A) \land (\exists U_2(p) \in \Tau)(U_2(p) \subseteq B)\\
&\implies (\exists U(p) \in \Tau)(U(p) \subseteq A \cup B) \text{,  e.g., } U(p) = U_1(p)\\
&\iff p \in \interior(A \cup B) \text{.}
\end{align*}
\item In a similar manner
\begin{align*}
p \in \interior(A \cap B) &\iff (\exists U(p) \in \Tau)(U(p) \subseteq A \cap B)\\
&\iff (\exists U(p) \in \Tau)(U(p) \subseteq A \land U(p) \subseteq B)\\
&\iff (\exists U(p) \in \Tau)(U(p) \subseteq A) \land (\exists U(p) \in \Tau)(U(p) \subseteq A)\\
&\iff p \in \mathring{A} \land p \in \mathring{B}\\
&\iff p \in \mathring{A} \cap \mathring{B}.\\
\end{align*}
\item This follows directly from the definition of a sets interior. 
\end{enumerate}
\end{proof}

\begin{lemma}
Let $(X, \Tau)$ be a topological space,  let $A,Y \subseteq X$, and let $p \in X$. Then 
\begin{equation}
(\exists U(p) \in \Tau_Y) \implies (p \in Y).
\end{equation}
\end{lemma}

\begin{proof}
We have, by definition of a relative topology, that $(\exists V \in \Tau)(U(p) = V \cap Y)$.  By definition of a sets intersection, we must have $p \in Y$.
\end{proof}
\begin{theorem}
Let $(X, \Tau)$ be a topological space, and let $Y \subseteq X$. Then 
\begin{enumerate}
\item $A \subseteq X \implies (\bar{A_Y} = \bar{A} \cap Y) \land (A'_Y = A' \cap Y)$
\item $\interior(A) \cap Y \subseteq \interior_Y (A)$
\item $\partial_Y A \subseteq \partial A \cap Y.$
\end{enumerate}
\end{theorem}

\begin{proof}
\begin{enumerate}
\item Suppose $A \subseteq X$. We have 
\begin{align*}
p \in \bar{A_Y} &\iff (\forall U(p) \in \Tau_Y)(U(p) \cap A \not = \emptyset)\\
&\iff (\forall V(p) \in \Tau)(V(p) \cap Y \cap A \not = \emptyset) \land p \in Y\text{, the first because } V(p) \cap Y \in \Tau_Y \text{ is a} \\
&\text{ neighborhood of } p \text{, and } p \in Y \text{ by Lemma 1}\\
&\iff (\forall V(p) \in \Tau)(V(p) \cap A \not = \emptyset) \land p \in Y\\
&\iff p \in \bar{A} \land p \in Y\\
&\iff p \in \bar{A} \cap Y.
\end{align*}
Thus,  $\bar{A_Y} = \bar{A} \cap Y$.  We also have
\begin{align*}
p \in A_Y' &\iff (\forall U(p) \in \Tau_Y)(U(p) \cap (A \backslash \{p\}) \not = \emptyset)\\
&\iff (\forall V(p) \in \Tau)(V(p) \cap Y \cap (A \backslash \{p\}) \not = \emptyset) \land p \in Y\text{, the first because } V(p) \cap Y \in \Tau_Y \text{ is a} \\
&\text{ neighborhood of } p \text{, and } p \in Y \text{ by Lemma 1}\\
&\iff (\forall V(p) \in \Tau)(V(p) \cap (A \backslash \{p\}) \not = \emptyset) \land p \in Y\\
&\iff p \in A' \land p \in Y\\
&\iff p \in A' \cap Y.
\end{align*}
Therefore, we have proven (1.).
\item We have 
\begin{align*}
p \in \interior(A) \cap Y &\iff (\exists U(p) \in \Tau)(U(p) \subseteq A) \land p \in Y\\
&\implies (\exists V(p) \in \Tau_Y)(V(p) \subseteq A), \text{ e.g., } V(p) = U(p) \cap Y\\
&\iff p \in \interior_Y(A) \text{,}
\end{align*}
which allows us to conclude that $\interior(A) \cap Y \subseteq \interior_Y (A)$.
\item Following as before
\begin{align*}
p \in \partial_Y A &\iff p \in \bar{A}_Y \cap \overline{\mathbb{C}A}_Y\\
&\iff (\forall U_1(p) \in \Tau_Y)(U_1(p) \cap A \not = \emptyset) \land (\forall U_2(p) \in \Tau_Y)(U_2(p) \cap \mathbb{C}A \not = \emptyset)\\
&\iff (\forall V_1(p) \in \Tau)(V_1(p) \cap A \not = \emptyset) \land (\forall V_2(p) \in \Tau)(V_2(p) \cap \mathbb{C}A \not = \emptyset) \land p \in Y, \text{ By}\\
&\text{ the same argument as in part (1.)}\\
&\iff p \in \bar{A} \land p \in \bar{\mathbb{C}A} \land p \in Y\\
&\iff p \in \partial A \cap Y
\end{align*}
and we have proven an even stronger condition of part (3.), 
\begin{align*}
\partial_Y A = \partial A \cap Y.
\end{align*}
\end{enumerate}
Finally,  we have completed the proof.
\end{proof}

DO NOT FORGET TO BRING UP 1.3 AND 2.3 IN CLASS
\end{document}