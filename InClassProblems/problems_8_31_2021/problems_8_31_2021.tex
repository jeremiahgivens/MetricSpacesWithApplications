\documentclass[12pt,a4paper]{article}
\usepackage[utf8]{inputenc}
\usepackage[a4paper,%
            left=1in,right=1in,top=1in,bottom=1in]{geometry}
\setlength{\headsep}{0.25in}

\usepackage{amsthm}

\usepackage{graphicx}
            
\usepackage[english]{babel}

\theoremstyle{theorem}
\newtheorem{theorem}{Theorem}
\newtheorem{lemma}{Lemma}
\newtheorem{corollary}{Corollary}
\newtheorem{case}{Case}

\theoremstyle{definition}
\newtheorem{definition}{Definition}

\usepackage{mathtools}
\DeclarePairedDelimiter\bra{\langle}{\rvert}
\DeclarePairedDelimiter\ket{\lvert}{\rangle}
\DeclarePairedDelimiterX\braket[2]{\langle}{\rangle}{#1 \delimsize\vert #2}

\usepackage{amsmath}
\usepackage{amsfonts}
\usepackage{amssymb}
\usepackage{fancyhdr}

\newcommand{\Tau}{\mathcal{T}}

\newenvironment{amatrix}[1]{%
  \left(\begin{array}{@{}*{#1}{c}|c@{}}
}{%
  \end{array}\right)
}

\usepackage{calligra}
\DeclareMathAlphabet{\mathcalligra}{T1}{calligra}{m}{n}
\DeclareFontShape{T1}{calligra}{m}{n}{<->s*[2.2]callig15}{}

\newcommand{\scripty}[1]{\ensuremath{\mathcalligra{#1}}}

\pagestyle{fancy}
\author{Jeremiah Givens}
\newcommand{\subject}{Metric Spaces}
\newcommand{\Date}{8/24/2021} 
\makeatletter
\rhead{{\small\@author}}
\lhead{{\small\subject}}
\chead{{\large In Class Exercises 8/31/2021}}
\cfoot{}
\rfoot{\thepage}
\lfoot{\today}

\renewcommand{\theequation}{\arabic{equation}}

\begin{document}
\begin{corollary}
Let $(X, \Tau)$ be a topological space, and $A \subseteq X$. Then 
\begin{equation*}
A \text{ is closed } \iff A = \bar{A}
\end{equation*}
\end{corollary}

\begin{proof}
By definition, we have $\bar{A} = A \cup A'$.  Also, if $A$ is closed, then $A' \subseteq A$. Thus,  we have
\begin{align*}
A \text{ is closed} &\iff A' \subseteq A\\
&\iff A \cup A' = A\\
&\iff \bar{A} = A\text{,}
\end{align*}
and our proof is complete.
\end{proof}

\begin{theorem}[(Closure)] Let $(X, \Tau)$ be a topological space, and let $A,B \subseteq X$. Then 
\begin{enumerate}
\item $A \subseteq B \implies \bar{A} \subseteq \bar{B}$
\item $\overline{A \cup B} = \bar{A} \cup \bar{B}$
\item $\overline{A \cap B} \subseteq \bar{A} \cap \bar{B}$
\item $\bar{A} = \bigcap \{F : A \subseteq F \land F \text{ is closed} \}$
\end{enumerate}
\end{theorem}

\begin{proof}
\begin{enumerate}
\item Suppose $A \subseteq B$. We have
\begin{align*}
p \in \bar{A} &\iff p \in A \cup A' \\
&\iff p \in B \cup A' &&\text{Since }A \subseteq B\\
&\iff p \in B \cup B' &&\text{Since }(\forall U)(U \cap (A \backslash \{p \}) \subseteq  U \cap (B \backslash \{p \}))\\
&\iff p \in \bar{B} \text{,}
\end{align*}
which proves that $\bar{A} \subseteq \bar{B}$.
\item We have
\begin{align*}
\overline{A \cup B} &= (A \cup B) \cup (A \cup B)'\\
&= (A \cup B) \cup \{p \in X | (\forall U(p))(U(p) \cap [(A \cup B)\backslash \{ p\}] \not = \emptyset \}.
\end{align*}
Now suppose for some $p \in X$,  there exists $U_1(p), U_2(p) \in \Tau$ such that $U_1(p) \cap [(A \cup B)\backslash \{ p\}] \not = \emptyset$,  $U_2(p) \cap [(A \cup B)\backslash \{ p\}] \not = \emptyset$,  $U_1(p) \cap (A \backslash \{ p\}) = \emptyset$,  and $U_2(p) \cap (B \backslash \{ p\}) = \emptyset$. Then, $p \in U_1(p) \cap U_2(p) = \{p\} \in \Tau$, but $\{p \} \cap [(A \cup B)\backslash \{ p\}]  = \emptyset$, which is a contradiction. Therefore, with all that we can conclude
\begin{align*}
\overline{A \cup B} &= (A \cup B) \cup \{p \in X | (\forall U(p))(U(p) \cap (A\backslash \{ p\} \not = \emptyset \lor U(p) \cap (B\backslash \{ p\}) \not = \emptyset \}\\
&= A \cup B) \cup \{p \in X | p \in A' \lor p \in B' \}\\
&= A \cup B \cup A' \cup B'\\
&= A \cup A' \cup B \cup B'\\
&= \bar{A} \cup \bar{B} \text{,}
\end{align*}
as desired.
\item In a similar manner
\begin{align*}
p \in \overline{A \cap B} &\iff p \in (A \cap B) \land  (\forall U(p))(U(p) \cap [(A \cap B)\backslash \{ p\}] \not = \emptyset \}\\
&\implies p \in (A \cap B) \land  (\forall U(p))(U(p) \cap (A \backslash \{ p\}) \not = \emptyset \land U(p) \cap (B \backslash \{ p\}) \not = \emptyset \}\\
&\iff p \in (A \cap B) \cup (A' \cap B')\\
&\iff p \in (A \cup A') \cap (B \cup B')\\
&\iff p \in \bar{A} \cap \bar{B} \text{,}
\end{align*}
which means that $\overline{A \cap B} \subseteq \bar{A} \cap \bar{B}$.
\end{enumerate}
\end{proof}
\end{document}