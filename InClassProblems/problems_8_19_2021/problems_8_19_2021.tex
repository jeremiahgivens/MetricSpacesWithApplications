\documentclass[12pt,a4paper]{article}
\usepackage[utf8]{inputenc}
\usepackage[a4paper,%
            left=1in,right=1in,top=1in,bottom=1in]{geometry}
\setlength{\headsep}{0.25in}

\usepackage{amsthm}

\usepackage{graphicx}
            
\usepackage[english]{babel}

\theoremstyle{theorem}
\newtheorem{theorem}{Theorem}

\theoremstyle{definition}
\newtheorem{definition}{Definition}

\usepackage{mathtools}
\DeclarePairedDelimiter\bra{\langle}{\rvert}
\DeclarePairedDelimiter\ket{\lvert}{\rangle}
\DeclarePairedDelimiterX\braket[2]{\langle}{\rangle}{#1 \delimsize\vert #2}

\usepackage{amsmath}
\usepackage{amsfonts}
\usepackage{amssymb}
\usepackage{fancyhdr}

\newenvironment{amatrix}[1]{%
  \left(\begin{array}{@{}*{#1}{c}|c@{}}
}{%
  \end{array}\right)
}

\usepackage{calligra}
\DeclareMathAlphabet{\mathcalligra}{T1}{calligra}{m}{n}
\DeclareFontShape{T1}{calligra}{m}{n}{<->s*[2.2]callig15}{}

\newcommand{\scripty}[1]{\ensuremath{\mathcalligra{#1}}}

\pagestyle{fancy}
\author{Jeremiah Givens}
\newcommand{\subject}{Metric Spaces}
\newcommand{\Date}{8/19/2021} 
\makeatletter
\rhead{{\small\@author}}
\lhead{{\small\subject}}
\chead{{\large Problems from Class \Date}}
\cfoot{}
\rfoot{\thepage}
\lfoot{\today}

\renewcommand{\theequation}{\arabic{equation}}

\begin{document}
\begin{theorem}
Let $A_n = \left[ 0, \frac{1}{n}\right] = \{ x \in \mathbb{R} : 0 \leq x \leq \frac{1}{n} \}$. Then, 
\begin{equation*}
\bigcap_{n \in \mathbb{N}} A_n = \{0 \} \text{.}
\end{equation*}
\end{theorem}
\begin{proof}
We have 
\begin{align*}
x \in \{0 \} &\implies x = 0 \\
&\implies (\forall n \in \mathbb{N})(0 \leq x \leq \frac{1}{n})\\
&\implies (\forall n \in \mathbb{N})(x \in A_n)\\
&\implies x \in \bigcap_{n \in \mathbb{N}} A_n \text{.}
\end{align*}
Therefore, 
\begin{equation*}
\{ 0 \} \subseteq \bigcap_{n \in \mathbb{N}} A_n.
\end{equation*}
Suppose
\begin{align*}
x \in \bigcap_{n \in \mathbb{N}} A_n \text{,}
\end{align*}
and for the sake of contradiction that $x \not \in \{ 0 \}$.
We can see
\begin{align*}
x \in \bigcap_{n \in \mathbb{N}} A_n \land x \not \in \{0 \} &\implies (\exists n \in \mathbb{N}) (x \in A_n \land x\not \in \{0 \})\\
&\implies (\exists n \in \mathbb{N}) (x \in A_n \land x \neq 0)\\
&\implies (\exists n \in \mathbb{N}) (0 < x \leq \frac{1}{n})\\
&\implies (\exists m \in \mathbb{N})(\frac{1}{m} < x)\\
&\implies (\exists m \in \mathbb{N})(x \not \in A_m)\\
&\implies x \not \in \bigcap_{n \in \mathbb{N}} A_n \text{,}
\end{align*}
which is a contradiction. Therefore, $x \in \{0 \}$, and we can conclude that 
\begin{equation*}
\bigcap_{n \in \mathbb{N}} A_n = \{0 \} \text{.}
\end{equation*}
\end{proof}

\begin{theorem}
Let $f: X \to Y$. Then
\begin{enumerate}
\item hello mate
\item yoooooo
\end{enumerate}
\end{theorem}
\end{document}