\documentclass[12pt,a4paper]{article}
\usepackage[utf8]{inputenc}
\usepackage[a4paper,%
            left=1in,right=1in,top=1in,bottom=1in]{geometry}
\setlength{\headsep}{0.25in}

\usepackage{amsthm}

\usepackage{graphicx}
            
\usepackage[english]{babel}

\theoremstyle{theorem}
\newtheorem{theorem}{Theorem}

\theoremstyle{definition}
\newtheorem{definition}{Definition}

\usepackage{mathtools}
\DeclarePairedDelimiter\bra{\langle}{\rvert}
\DeclarePairedDelimiter\ket{\lvert}{\rangle}
\DeclarePairedDelimiterX\braket[2]{\langle}{\rangle}{#1 \delimsize\vert #2}

\usepackage{amsmath}
\usepackage{amsfonts}
\usepackage{amssymb}
\usepackage{fancyhdr}

\newenvironment{amatrix}[1]{%
  \left(\begin{array}{@{}*{#1}{c}|c@{}}
}{%
  \end{array}\right)
}

\usepackage{calligra}
\DeclareMathAlphabet{\mathcalligra}{T1}{calligra}{m}{n}
\DeclareFontShape{T1}{calligra}{m}{n}{<->s*[2.2]callig15}{}

\newcommand{\scripty}[1]{\ensuremath{\mathcalligra{#1}}}

\pagestyle{fancy}
\author{Jeremiah Givens}
\newcommand{\subject}{Metric Spaces}
\newcommand{\Date}{8/19/2021} 
\makeatletter
\rhead{{\small\@author}}
\lhead{{\small\subject}}
\chead{{\large Problems from Class \Date}}
\cfoot{}
\rfoot{\thepage}
\lfoot{\today}

\renewcommand{\theequation}{\arabic{equation}}

\begin{document}
\begin{theorem}
Let $A_n = \left[ 0, \frac{1}{n}\right] = \{ x \in \mathbb{R} : 0 \leq x \leq \frac{1}{n} \}$. Then, 
\begin{equation*}
\bigcap_{n \in \mathbb{N}} A_n = \{0 \} \text{.}
\end{equation*}
\end{theorem}
\begin{proof}
We have 
\begin{align*}
x \in \{0 \} &\implies x = 0 \\
&\implies (\forall n \in \mathbb{N})(0 \leq x \leq \frac{1}{n})\\
&\implies (\forall n \in \mathbb{N})(x \in A_n)\\
&\implies x \in \bigcap_{n \in \mathbb{N}} A_n \text{.}
\end{align*}
Therefore, 
\begin{equation*}
\{ 0 \} \subseteq \bigcap_{n \in \mathbb{N}} A_n.
\end{equation*}
Suppose
\begin{align*}
x \in \bigcap_{n \in \mathbb{N}} A_n \text{,}
\end{align*}
and for the sake of contradiction that $x \not \in \{ 0 \}$.
We can see
\begin{align*}
x \in \bigcap_{n \in \mathbb{N}} A_n \land x \not \in \{0 \} &\implies (\exists n \in \mathbb{N}) (x \in A_n \land x\not \in \{0 \})\\
&\implies (\exists n \in \mathbb{N}) (x \in A_n \land x \neq 0)\\
&\implies (\exists n \in \mathbb{N}) (0 < x \leq \frac{1}{n})\\
&\implies (\exists m \in \mathbb{N})(\frac{1}{m} < x)\\
&\implies (\exists m \in \mathbb{N})(x \not \in A_m)\\
&\implies x \not \in \bigcap_{n \in \mathbb{N}} A_n \text{,}
\end{align*}
which is a contradiction. Therefore, $x \in \{0 \}$, and we can conclude that 
\begin{equation*}
\bigcap_{n \in \mathbb{N}} A_n = \{0 \} \text{.}
\end{equation*}
\end{proof}

\begin{theorem}
Let $f: X \to Y$. Then
\begin{enumerate}
\item $f(A \cup B) = f(A) \cup f(B)$
\item $A \subseteq B \implies f(A) \subseteq f(B)$
\item $f^{-1}(C \cup D) = f^{-1}(C) \cup f^{-1}(D)$
\item $f^{-1}(C \cap D) = f^{-1}(C) \cap f^{-1}(D)$
\item $f(A \cap B) \subseteq f(A) \cap f(B)$
\item $f^{-1}(\mathbb{C}A) = \mathbb{C} (f^{-1}(A))$
\item $f(\mathbb{C}A) ??? \mathbb{C}(f(A))$ (this is an open question)
\end{enumerate}
\end{theorem}

\begin{proof}
\begin{enumerate}
\item Applying the definition of a set's image and the definition of unions of sets,  we see
\begin{align*}
y \in f(A \cup B) &\iff (\exists x \in A \cup B)(f(x) = y)\\
&\iff (\exists x \in A)(f(x) = y) \lor (\exists x \in B)(f(x) = y)\\
&\iff y \in f(A) \lor y \in f(B)\\
&\iff y \in f(A) \cup f(B) \text{.}
\end{align*}
From this,  we can conclude that $f(A \cup B) = f(A) \cup f(B)$.

\item Suppose $A \subseteq B$.  We have
\begin{align*}
y \in f(A) &\iff (\exists x \in A)(f(x) = y) && \\
&\implies (\exists x \in B) (f(x) = y) && \text{By initial assumption}\\
&\iff y \in f(B) \text{,}
\end{align*}
which means $f(A) \subseteq f(B)$.

\item By definition of a set's preimage and the definition of a union of sets, we have
\begin{align*}
x \in f^{-1}(C \cup D) &\iff f(x) \in C \cup D \\
&\iff f(x) \in C \lor f(x) \in D \\
&\iff x \in f^{-1}(C) \lor x \in f^{-1}(D)\\
&\iff x \in f^{-1}(C) \cup f^{-1}(D) \text{.}
\end{align*}
Therefore,  $f^{-1}(C \cup D) = f^{-1}(C) \cup f^{-1}(D)$.
\item From the definition of a set's preimage and the definition of an intersection of two sets, 
\begin{align*}
x \in f^{-1}(C \cap D) &\iff f(x) \in C \cap D \\
&\iff f(x) \in C \land f(x) \in D \\
&\iff x \in f^{-1}(C) \land x \in f^{-1}(D)\\
&\iff x \in f^{-1}(C) \cap f^{-1}(D) \text{.}
\end{align*}
Thus,  $f^{-1}(C \cap D) = f^{-1}(C) \cap f^{-1}(D)$.
\item From the definition of a set's image and the definition of an intersection of sets,
\begin{align*}
y \in f(A \cap B) &\iff (\exists x \in A \cap B)(f(x) = y)\\
&\iff (\exists x)(x \in A \land x \in B \land f(x) = y)\\
&\implies (\exists x \in A)(f(x) = y) \land (\exists x \in B)(f(x) = y)\\
&\iff y \in f(A) \land y \in f(B)\\
&\iff y \in f(A) \cap f(B) \text{.}
\end{align*}
Thus,  $f(A \cap B) \subseteq f(A) \cap f(B)$.
\item From the definitions of a set's preimage and the complement of a set, we have
\begin{align*}
x \in f^{-1}(\mathbb{C} A) &\iff f(x) \in \mathbb{C} A \\
&\iff f(x) \in Y \land f(x) \not \in A\\
&\iff x \in f^{-1}(Y) \land x \not \in f^{-1}(A)\\
&\iff x \in \mathbb{C} f^{-1}(A) \text{.}
\end{align*}
Therefore,  $f^{-1}(\mathbb{C}A) = \mathbb{C} (f^{-1}(A))$.

\item Expanding the sets by definition,  can see
\begin{align*}
y \in \mathbb{C} f(A) &\iff y \in f(X) \land y \not \in f(A) \\
&\iff (\exists x \in X)(y = f(x) \land y \not \in f(A))\\
&\iff (\exists x \in X)(f(x) \not \in f(A) \land f(x) = y)\\
&\implies (\exists x \in X)(x \not \in A \land f(x) = y)\\
&\iff (\exists x \in \mathbb{C}A)(f(x) = y)\\
&\iff y \in f(\mathbb{C}A) \text{,}
\end{align*}
which leads us to conclude that $\mathbb{C} f(A) \subseteq f(\mathbb{C}A)$.
\end{enumerate}
Finally,  our proof is complete.
\end{proof}
\end{document}