\documentclass[10pt,a4paper]{article}
\usepackage[utf8]{inputenc}
\usepackage[a4paper,%
            left=1in,right=1in,top=1in,bottom=1in]{geometry}
\setlength{\headsep}{0.25in}

\usepackage{amsthm}

\usepackage{graphicx}
            
\usepackage[english]{babel}

\theoremstyle{theorem}
\newtheorem{theorem}{Theorem}
\newtheorem{prop}{Proposition}
\newtheorem{CA}{Counter Example}
\newtheorem{lemma}{Lemma}
\newtheorem{corollary}{Corollary}
\newtheorem{case}{Case}

\theoremstyle{definition}
\newtheorem{definition}{Definition}

\usepackage{mathtools}
\DeclarePairedDelimiter\bra{\langle}{\rvert}
\DeclarePairedDelimiter\ket{\lvert}{\rangle}
\DeclarePairedDelimiterX\braket[2]{\langle}{\rangle}{#1 \delimsize\vert #2}

\usepackage{amsmath}
\usepackage{amsfonts}
\usepackage{amssymb}
\usepackage{fancyhdr}

\DeclareMathOperator{\interior}{int}

\newcommand{\Tau}{\mathcal{T}}

\newenvironment{amatrix}[1]{%
  \left(\begin{array}{@{}*{#1}{c}|c@{}}
}{%
  \end{array}\right)
}

\usepackage{calligra}
\DeclareMathAlphabet{\mathcalligra}{T1}{calligra}{m}{n}
\DeclareFontShape{T1}{calligra}{m}{n}{<->s*[2.2]callig15}{}

\newcommand{\scripty}[1]{\ensuremath{\mathcalligra{#1}}}

\pagestyle{fancy}
\author{Jeremiah Givens}
\newcommand{\subject}{Metric Spaces}
\newcommand{\Date}{9/2/2021} 
\makeatletter
\rhead{{\small\@author}}
\lhead{{\small\subject}}
\chead{{\large In Class Exercise 9/7/2021}}
\cfoot{}
\rfoot{\thepage}
\lfoot{\today}

\renewcommand{\theequation}{\arabic{equation}}

\begin{document}
\begin{prop}
Let $(X, \Tau)$ be a topological space, and let $Y \subseteq X$. Then 
\begin{equation*}
A \subseteq X \implies \bar{A_Y} = \bar{A} \cap Y.
\end{equation*}
\end{prop}

\begin{CA}
Let $X = \{1, 2\}$, and $\Tau = \{\emptyset, X\}$. Also, let $Y = \{1\}$, and $A = \{2\}$. Then,  we have $A \subseteq X$,  $\Tau_Y = \{\emptyset, Y\}$, $\bar{A} = \{2\}$, $\bar{A}_Y = \emptyset$, and $\bar{A} \cap Y = \emptyset$. Thus, $\bar{A}_Y \not = \bar{A} \cap Y$.

I believe the issue comes from the statement a student made when you were first writing the proof. The student set that $A$ only need be a subset of $X$, where you had originally wrote $A \subset Y$. I believe the way you had it originally is the necessary condition for this implication to hold. With this, my original proof works out:
\end{CA}

\begin{theorem}
Let $(X, \Tau)$ be a topological space, and let $Y \subseteq X$. Then 
\begin{equation*}
A \subseteq Y \implies (\bar{A_Y} = \bar{A} \cap Y) \land (A'_Y = A' \cap Y).
\end{equation*}
\end{theorem}

\begin{proof}
Suppose $A \subseteq Y$.  Then 
\begin{align*}
p \in \bar{A_Y} &\iff (\forall U(p) \in \Tau_Y)(U(p) \cap A \not = \emptyset)\\
&\iff (\forall V(p) \in \Tau)(V(p) \cap Y \cap A \not = \emptyset) \land p \in Y\text{,  because} V(p) \cap Y \in \Tau_Y \text{ is a neighborhood of } p\\
&\iff (\forall V(p) \in \Tau)(V(p) \cap A \not = \emptyset) \land p \in Y\text{,  because } A \subseteq Y \implies V(p) \cap Y \cap A = V(p) \cap A\\
&\iff p \in \bar{A} \land p \in Y\\
&\iff p \in \bar{A} \cap Y.
\end{align*}
Thus,  $\bar{A_Y} = \bar{A} \cap Y$.  
\end{proof}
\end{document}