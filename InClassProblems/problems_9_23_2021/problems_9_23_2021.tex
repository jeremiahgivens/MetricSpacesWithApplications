\documentclass[10pt,a4paper]{article}
\usepackage[utf8]{inputenc}
\usepackage[a4paper,%
            left=.75in,right=.75in,top=1in,bottom=1in]{geometry}
\setlength{\headsep}{0.25in}

\usepackage{amsthm}

\usepackage{graphicx}
            
\usepackage[english]{babel}

\theoremstyle{theorem}
\newtheorem{theorem}{Theorem}
\newtheorem{lemma}{Lemma}
\newtheorem{corollary}{Corollary}
\newtheorem{case}{Case}

\theoremstyle{definition}
\newtheorem{definition}{Definition}

\usepackage{mathtools}
\DeclarePairedDelimiter\bra{\langle}{\rvert}
\DeclarePairedDelimiter\ket{\lvert}{\rangle}
\DeclarePairedDelimiterX\braket[2]{\langle}{\rangle}{#1 \delimsize\vert #2}

\usepackage{amsmath}
\usepackage{amsfonts}
\usepackage{amssymb}
\usepackage{fancyhdr}

\DeclareMathOperator{\interior}{int}

\newcommand{\Tau}{\mathcal{T}}

\newenvironment{amatrix}[1]{%
  \left(\begin{array}{@{}*{#1}{c}|c@{}}
}{%
  \end{array}\right)
}

\usepackage{calligra}
\DeclareMathAlphabet{\mathcalligra}{T1}{calligra}{m}{n}
\DeclareFontShape{T1}{calligra}{m}{n}{<->s*[2.2]callig15}{}

\newcommand{\scripty}[1]{\ensuremath{\mathcalligra{#1}}}

\pagestyle{fancy}
\author{Jeremiah Givens}
\newcommand{\subject}{Metric Spaces}
\newcommand{\Date}{9/2/2021} 
\makeatletter
\rhead{{\small\@author}}
\lhead{{\small\subject}}
\chead{{\large In Class Problems 9/16/2021}}
\cfoot{}
\rfoot{\thepage}
\lfoot{\today}

\renewcommand{\theequation}{\arabic{equation}}

\begin{document}
\begin{theorem}
Let $X$ and $Y$ be topological spaces, and let $f:X \to Y$ be a continuous bijection.  Then $f$ is a homeomorphism iff either 
\begin{enumerate}
\item $f$ is open
\item $f$ is closed.
\end{enumerate}
\end{theorem}
\begin{proof}
Suppose $f$ is a homeomorphism. Let $G \in \Tau_X$. Then,  the inverse of $f^{-1}$ is continuous, and 
\begin{align*}
f(G) &= (f^{-1})^{-1}(G)
\end{align*}
which is open, by definition. The argument works if you switch open for closed in the above.

Suppose now that $f$ is open. Let $X \in \Tau_X$.  Then $f(X) = (f^{-1})^{-1}(X)$ is open in $Y$.  Therefore, $f$ has a continuous inverse, and is therefore a homeomorphism. Swapping open for closed in the above argument will complete our proof.
\end{proof}

\begin{theorem}
Let $\{X_k\}_{k = 1}^n$ be a finite collection of topological spaces, and let $X$ be a topological space. Let $f:X \to \Pi_{k=1}^n X_k$. Then the following are equivalent:
\begin{enumerate}
\item $f$ is continuous
\item $p_j \circ f$ is continuous for all $j \in \{1, ..,n\}$.
\end{enumerate}
\end{theorem}

\begin{proof}
$(1) \implies (2)$. Suppose that $f$ is continuous. We have by definition that $p_j$ is continuous. Let $U_j \in \Tau_{X_j}$. Then, since $p_j$ is continuous, $p_{j}^{-1}(U_j)$ is open in $\Pi_{k=1}^n X_k$. Then, since $f$ is continuous, $f^{-1}(p_{j}^{-1}(U_j)) \in \Tau_X$.  Since $f^{-1}(p_{j}^{-1}(U_j)) = (p_j \circ f)^{-1}(U_j)$, we can conclude that $p_j \circ f$ is continuous for all $j \in \{1, ..,n\}$.

$(2) \implies (1)$. Now suppose $p_j \circ f$ is continuous for all $j \in \{1, ..,n\}$. By one of our many continuity theorems, we have that $f$ is continuous if and only if $f^{-1}(U)$ is open for all members $U$ of a subbasis of  $\Pi_{k=1}^n X_k$. Let $U_j \in \Tau_j$. Then, $p^{-1}(U_j)$ is, by definition of the product topology, a member of a subbasis of $\Pi_{k=1}^n X_k$. We have
\begin{align*}
f^{-1}(p^{-1}(U_j)) &= (p_j \circ f)^{-1}(U_j),
\end{align*}
which by initial assumption we know is continuous.
\end{proof}
\end{document}