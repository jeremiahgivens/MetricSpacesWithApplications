\documentclass[10pt,a4paper]{article}
\usepackage[utf8]{inputenc}
\usepackage[a4paper,%
            left=.75in,right=.75in,top=1in,bottom=1in]{geometry}
\setlength{\headsep}{0.25in}

\usepackage{amsthm}

\usepackage{graphicx}
\usepackage{pgfplots}
            
\usepackage[english]{babel}

\theoremstyle{theorem}
\newtheorem{theorem}{Theorem}
\newtheorem{lemma}{Lemma}
\newtheorem{corollary}{Corollary}
\newtheorem{case}{Case}

\newcommand\restr[2]{{% we make the whole thing an ordinary symbol
  \left.\kern-\nulldelimiterspace % automatically resize the bar with \right
  #1 % the function
  \vphantom{\big|} % pretend it's a little taller at normal size
  \right|_{#2} % this is the delimiter
  }}

\theoremstyle{definition}
\newtheorem{definition}{Definition}
\newtheorem{remark}{Remark}

\usepackage{mathtools}
\DeclarePairedDelimiter\bra{\langle}{\rvert}
\DeclarePairedDelimiter\ket{\lvert}{\rangle}
\DeclarePairedDelimiterX\braket[2]{\langle}{\rangle}{#1 \delimsize\vert #2}

\usepackage{amsmath}
\usepackage{amsfonts}
\usepackage{amssymb}
\usepackage{fancyhdr}

\DeclareMathOperator{\interior}{int}

\newcommand{\Tau}{\mathcal{T}}

\newenvironment{amatrix}[1]{%
  \left(\begin{array}{@{}*{#1}{c}|c@{}}
}{%
  \end{array}\right)
}

\usepackage{calligra}
\DeclareMathAlphabet{\mathcalligra}{T1}{calligra}{m}{n}
\DeclareFontShape{T1}{calligra}{m}{n}{<->s*[2.2]callig15}{}

\newcommand{\scripty}[1]{\ensuremath{\mathcalligra{#1}}}

\pagestyle{fancy}
\author{Jeremiah Givens}
\newcommand{\subject}{Metric Spaces}
\newcommand{\Date}{9/2/2021} 
\makeatletter
\rhead{{\small\@author}}
\lhead{{\small\subject}}
\chead{{\large Picard's Existence and Uniqueness Theorem}}
\cfoot{}
\rfoot{\thepage}
\lfoot{\today}

\renewcommand{\theequation}{\arabic{equation}}

\begin{document}
\begin{titlepage}
\vspace*{\fill}
\begin{center}
{\Huge Picard's Existence and Uniqueness Theorem}\\
Presenter: Jeremiah Givens\\
Professor: Dr. Claudio Morales\\
Deparment of Mathematics, University of Alabama in Huntsville
\end{center}
\vspace*{\fill}
\end{titlepage}

\section{Preliminaries}
\begin{lemma}
Let $x: [a, b] \to \mathbb{R}$ be continuous. Then, $x$ is bounded, and $x$ achieves it's maximum.
\end{lemma}

\begin{proof}
We have
\begin{align*}
[a, b] \text{ is compact} &\implies x([a, b]) \text{ is compact} && \text{continuous image of compact set}\\
&\implies x([a, b]) \text{ is closed and bounded} && \text{by the Heine-Borel Theorem, since } x([a, b]) \subseteq \mathbb{R}\\
&\implies x \text{ is bounded}.
\end{align*}

Now we need to show that $x$ achieves its maximum.  We know from our argument above, that $x$ has an upper bound. Thus, the least upper bound property of the real numbers tells us that the supremum $M = \sup \{x(t): t \in [a, b] \}$ exists. By the definition of a supremum, we can choose a sequence $\{t_n \}_{n \in \mathbb{N}}$ such that $M- \frac{1}{n} \leq x(t_n) \leq M$. Clearly, the sequence $\{x(t_n)\}_{n \in \mathbb{N}}$ converges to $M$. Since $[a, b]$ is compact, and therefore sequentially compact, we have that $\{t_n \}_{n \in \mathbb{N}}$ has a subsequence $\{t_{n_i} \}_{i \in \mathbb{N}}$ that converges to some $t \in [a, b]$. Since $x$ is continuous, $\{x(t_{n_i}) \}_{i \in \mathbb{N}}$ converges to $x(t)$. Since a subsequence of any convergent sequence converges to the same limit, we have that $\{x(t_{n_i}) \}_{i \in \mathbb{N}}$ converges to $M$. Therefore,  $x$ achieves its maximum at $t$.
\end{proof}

\begin{lemma}
Let $C$ equal the set of all continuous real-valued functions on the closed interval $[a, b]$. Then, $C$ forms a metric space with the metric $d$ defined by
\begin{align*}
d(x, y) = \max_{t \in [a, b]} |x(t) - y(t)|.
\end{align*}
\end{lemma}

\begin{proof}
Since we know every norm induces a metric, it will suffice to show that $|| \cdot ||: C \to \mathbb{R}$ defined by
\begin{align*}
||x|| = \max_{t \in [a, b]} |x(t)|
\end{align*}
forms a norm on $C$. Since the absolute value function is continuous, and the composition of two continuous functions is continuous, we can conclude from Lemma 1 that $|| x ||$ is well defined for all $x \in C$.

Let $s \in \mathbb{R}$, and $x \in C$. We have
\begin{align*}
||s x|| &= \max_{t \in [a, b]} |s x(t)|\\
&= \max_{t \in [a, b]} |s|| x(t)|\\
&= |s|\max_{t \in [a, b]} | x(t)| && \text{since } s \text{ has no dependence on } t\\
&= |s|||x||,
\end{align*}
and we have shown that $|| \cdot ||$ obeys the absolute homogeneity property of a norm.

Next, we must show that for all $x \in C$, $||x|| = 0 \implies x = 0$. Let $x \in C$. Proving the contrapositive, we have 
\begin{align*}
x \not = 0 &\implies (\exists t_0 \in [a, b])(x(t_0) \not = 0)\\
&\implies \max_{t \in [a, b]} |x(t)| \geq |x(t_0)| && \text{by definition of a max}\\
&\implies \max_{t \in [a, b]} |x(t)| > 0\\
&\implies ||x|| > 0.
\end{align*}

Finally, we must show that the triangle inequality holds for $|| \cdot ||$. Let $x, y \in C$. By Lemma 1, we have that $|x|$, $|y|, $ and $|x + y|$ achieve their maximums on $[a, b]$. Thus, we can define $t_1,t_2,t_3 \in [a, b]$ to be such that $|x(t_1)| = ||x||$, $|y(t_2)| = ||y||$, and $|x(t_3) + y(t_3)| = ||x + y||$. With this, we have
\begin{align*}
||x + y|| &= |x(t_3) + y(t_3)|\\
&\leq |x(t_3)| + |y(t_3)| && \text{Since the absolute value function forms a norm on the real numbers}\\
&\leq |x(t_1)| + |y(t_3)| && \text{Since } |x(t_1)| \geq |x(t_3)| \text{ by definition of a max}\\
&\leq |x(t_1)| + |y(t_2)| && \text{Since } |y(t_2)| \geq |y(t_3)| \text{ by definition of a max}\\
&= ||x|| + ||y||,
\end{align*}
and our proof is complete.
\end{proof}

\begin{lemma}
The metric space $(C, d)$ from the previous lemma is complete.
\end{lemma}

\begin{proof}
Let $\{x_n\}_{n \in \mathbb{N}}$ be a Cauchy sequence in $C$.  Define $f:[a, b] \to \mathbb{R}$ by $f(t) = \lim_{n \to \infty} x_n(t)$. We need to show that $f$ is well defined, that $f$ is continuous, and that $f = \lim_{n \to \infty} x_n$.

To show that $f$ is well defined, it will suffice to show that for all $t \in [a, b]$, the sequence $\{x_n(t) \}_{n=1}^\infty$ is Cauchy, since $\mathbb{R}$ is complete. Let $\epsilon > 0$. Since $\{x_n\}_{n \in \mathbb{N}}$ is Cauchy, there exists an $N \in \mathbb{N}$ such that for all $n, m \geq N$,
\begin{align*}
||x_n - x_m|| < \epsilon &\iff \max_{t \in [a, b]} |x_n(t) - x_m(t)| < \epsilon\\
&\implies (\forall t \in [a, b])(|x_n(t) - x_m(t)| < \epsilon).
\end{align*}
From this, we can conclude that $f$ is well defined.

We will now show that $f$ is continuous. Let $\epsilon >0$, and let $p_0 \in [a, b]$. Since $\{x_n(p)\}_{n \in \mathbb{N}}$ converges to $f(p)$ for all $p \in [a, b]$, there exists an $n \in \mathbb{N}$ such that $|f(p) - x_{n}(p)| < \frac{\epsilon}{3}$. Since $x_n$ is continuous at $p_0$, there exists a $\delta > 0$ such that for all $p \in [a, b]$
\begin{align*}
|p - p_0| < \delta \implies |x_n(p) - x_n(p_0)| < \frac{\epsilon}{3}.
\end{align*} 
With this, we have that for all $p \in [a, b]$ with $|p - p_0| < \delta$
\begin{align*}
|f(p) - f(p_0)| &\leq |f(p) - x_{n}(p)| + |f(p_0) - x_{n}(p)| &&\text{triangle inequality}\\
&\leq |f(p) - x_{n}(p)| + |f(p_0) - x_{n}(p_0)| + |x_n(p_0) - x_{n}(p)| &&\text{triangle inequality} \\
&< \frac{\epsilon}{3} + \frac{\epsilon}{3} + \frac{\epsilon}{3}\\
&= \epsilon.
\end{align*}
Thus, we have that $f$ is continuous, and therefore $f \in C$.

Finally, all that remains is to show that $f = \lim_{n \to \infty} x_n$. Let $\epsilon >0$. Since $\{x_n(p)\}_{n \in \mathbb{N}}$ is Cauchy, there exists an $N \in \mathbb{N}$ such that for all $m, n \geq N$, we have $||x_n - x_m|| < \epsilon$. Then, for all $t \in [a, b]$ and $m \geq N$, we have
\begin{align*}
|f(t) - x_m(t)| &= |\lim_{n \to \infty} x_n(t) - x_m(t)| && \text{by definition of } f\\
&= \lim_{n \to \infty}|x_n(t) - x_m(t)| && \text{by elemetary limit laws}\\
&\leq \lim_{n \to \infty}||x_n - x_m|| && \text{by definition of our norm}\\
&\leq \epsilon. && \text{by definition of } N \text{ above}
\end{align*}
Thus, since $\epsilon$ was arbitrary, we have shown that $f = \lim_{n \to \infty} x_n$, and that $C$ is complete.
\end{proof}

\begin{lemma}
Let $x_0 \in \mathbb{R}$, and let $c \in (0, \infty)$. Let $\tilde{C}$ be a subspace of the metric space from Lemma 1, consisting of all functions $x \in C$ such that 
\begin{align*}
d(x, x_0) \leq c.
\end{align*}
Then, $\tilde{C}$ is closed.
\end{lemma}

\begin{proof}
Let $f \in \bar{\tilde{C}}$.  We have
\begin{align*}
f \in \bar{\tilde{C}} &\iff (\forall \epsilon > 0)(B(f; \epsilon) \cap \tilde{C} \not = \emptyset)\\
&\iff (\forall \epsilon > 0)(\exists x \in \tilde{C})(d(f, x) < \epsilon)\\
&\iff (\forall n \in \mathbb{N})(\exists x_n \in \tilde{C})(d(f, x_n) < \frac{1}{n}).
\end{align*}
With this, we have
\begin{align*}
d(f, x_0) &\leq d(f, x_n) + d(x_n, x_0) && \text{triangle inequality}\\
&< \frac{1}{n} + d(x_n, x_0)\\
&\leq \frac{1}{n} + c && \text{since } x_n \in \tilde{C}\\
\end{align*}
Since this is true for all $n \in \mathbb{N}$, we have $d(f, x_0) \leq c$ which implies $f \in \tilde{C}$.
\end{proof}

\begin{lemma}
Let $C$ be a complete metric space. Let $A \subseteq C$ be closed. Then, $A$ is complete.
\end{lemma}

\begin{proof}
Let $A$ be closed, and let $\{x_n\}_{n \in \mathbb{N}}$ be a Cauchy sequence in $A$.  Since $X$ is complete, we know that the limit converges to some point $x$ in $X$.  We have
\begin{align*}
\lim_{n \to \infty} x_n = x &\implies (\forall \epsilon > 0)(\exists N \in \mathbb{N})(n \geq N \implies d(x_n, x) \leq \epsilon)\\
&\implies (\forall \epsilon > 0)(\exists x_n \in A)(x_n \in B(x; \epsilon))\\
&\implies (\forall \epsilon > 0)(B(x; \epsilon) \cap A \not = \emptyset)\\
&\implies x \in \bar{A}\\
&\implies x \in A. && \text{since } A \text{ is closed}
\end{align*}
Since this is true for any Cauchy sequence in $A$, $A$ is complete.
\end{proof}

\section{Main Topic}
The focus of this paper is on solutions to explicit ordinary differential equations of the form
\begin{align}
x' = f(t, x)
\end{align}
where $x: \mathbb{R} \to \mathbb{R}$, $f: \mathbb{R} \times \mathbb{R} \to \mathbb{R}$, and the prime denotes differentiation of $x$ with respect to $t$.  Specifically, we are interested in initial value problems, where
\begin{align*}
x(t_0) = x_0.
\end{align*}

\begin{theorem}
Let $f$ be continuous on a rectangle 
\begin{align*}
R = \{(t, x) \mathbb \in {R}^2 : |t - t_0| \leq a \land |x - x_0| \leq b \},
\end{align*}
and thus bounded on $R$, say
\begin{align}
|f(t, x)| \leq c, \text{ for all } (t, x) \in R.
\end{align}
Suppose $f$ satisfies a Lipschitz condition on $R$ with respect to it's second argument, that is, there is a constant $k$ (Lipschitz constant) such that for $(t, x), (t, v) \in R$
\begin{align}
|f(t, x) - f(t, v)| \leq k |x - v|.
\end{align}
Then the initial value problem (1) has a unique solution. This solution exists on an interval $[t_0 - \beta, t_0 + \beta]$ where 
\begin{align}
\beta < \min \Bigg \{a, \frac{b}{c}, \frac{1}{k} \Bigg \}
\end{align}
\end{theorem}

\begin{proof}
Let $C(J)$ be the metric space of all real-valued continuous functions on the interval $J = [t_0 - \beta, t_0 + \beta]$ with metric $d$ defined by 
\begin{align*}
d(x, y) = \max_{t \in J} |x(t) - y(t)|.
\end{align*}
By Lemma 2 we know that $C(J)$ is a metric space, and by Lemma 3 we know that $C(J)$ is complete.  Let $\tilde{C}$ be a subspace of $C(J)$ consisting of all functions $x \in C(J)$ such that 
\begin{align}
|x(t) - x_0| \leq c \beta.
\end{align}
By Lemma 4,  we have that $\tilde{C}$ is closed, so by Lemma 5 $\tilde{C}$ is complete.

From the fundamental theorem of calculus, (1) can be written in the form $x = Tx$, where $T: \tilde{C} \to \tilde{C}$ is defined by 
\begin{align}
Tx(t) = x_0 + \int_{t_0}^{t} f(\tau, x(\tau))d \tau
\end{align}
We have $\tau \in J \implies |\tau - t_0| \leq a$ (by 4), and 
\begin{align*}
x \in \tilde{C} &\implies |x(\tau) - x_0| \leq c \beta &&\text{by (5)}\\
&\implies |x(\tau) - x_0| \leq b &&\text{by (4)}.
\end{align*}
Thus, $(\tau, x(\tau)) \in R$ and the integral in (6) exists since $f$ is continuous on $R$. To see that $T$ maps $\tilde{C}$ into itself,
\begin{align*}
|Tx(t) - x_0| &= \left|\int_{t_0}^{t} f(\tau, x(\tau))d \tau \Large\right|\\
&\leq c|t - t_0| && \text{by (2)}\\
&\leq c \beta && \text{since } t \in [t_0 - \beta, t_0 + \beta].
\end{align*}
Thus, by $(5)$ we have that $Tx \in \tilde{C}$.

We will now show that $T$ is a contraction on $\tilde{C}$. We have
\begin{align*}
|Tx(t) - Tv(t)| &= \left|\int_{t_0}^{t} f(\tau, x(\tau)) - f(\tau, v(\tau))d \tau \Large\right|\\
&\leq |t - t_0| \max_{\tau \in J} k|x(\tau) - v(\tau)| && \text{by the Lipschitz condition (3)}\\
&\leq k \beta d(x, v) && \text{by (4) and by definition of a the metric } d.
\end{align*}
Since the right hand side does not depend on $t$, we can take the max on both sides to yield 
\begin{align*}
d(Tx, Tv) \leq k \beta d(x, v).
\end{align*}
By (4), we have that $k \beta < 1$. Thus, we have shown that $T$ is a contraction mapping on $\tilde{C}$.

By the Banach Fixed Point Theorem, $T$ has a unique fixed point $x \in \tilde{C}$, such that $Tx = x$. That is, there exists a unique $x \in \tilde{C}$ such that 
\begin{align}
x(t) = x_0 + \int_{t_0}^{t} f(\tau, x(\tau))d \tau.
\end{align}
Since $(\tau, x(\tau)) \in R$ where $f$ is continuous, we have that both sides of (7) are differentiable,  and $x$ satisfies (1).  Since every solution $x$ to $(1)$ must satisfy $(7)$,  we have that $x$ is unique, and our proof is complete.
\end{proof}
\end{document}

