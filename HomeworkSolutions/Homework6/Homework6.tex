\documentclass[10pt,a4paper]{article}
\usepackage[utf8]{inputenc}
\usepackage[a4paper,%
            left=.75in,right=.75in,top=1in,bottom=1in]{geometry}
\setlength{\headsep}{0.25in}

\usepackage{amsthm}

\usepackage{graphicx}
            
\usepackage[english]{babel}

\theoremstyle{theorem}
\newtheorem{theorem}{Theorem}
\newtheorem{lemma}{Lemma}
\newtheorem{corollary}{Corollary}
\newtheorem{case}{Case}

\newcommand\restr[2]{{% we make the whole thing an ordinary symbol
  \left.\kern-\nulldelimiterspace % automatically resize the bar with \right
  #1 % the function
  \vphantom{\big|} % pretend it's a little taller at normal size
  \right|_{#2} % this is the delimiter
  }}

\theoremstyle{definition}
\newtheorem{definition}{Definition}
\newtheorem{remark}{Remark}

\usepackage{mathtools}
\DeclarePairedDelimiter\bra{\langle}{\rvert}
\DeclarePairedDelimiter\ket{\lvert}{\rangle}
\DeclarePairedDelimiterX\braket[2]{\langle}{\rangle}{#1 \delimsize\vert #2}

\usepackage{amsmath}
\usepackage{amsfonts}
\usepackage{amssymb}
\usepackage{fancyhdr}

\DeclareMathOperator{\interior}{int}

\newcommand{\Tau}{\mathcal{T}}

\newenvironment{amatrix}[1]{%
  \left(\begin{array}{@{}*{#1}{c}|c@{}}
}{%
  \end{array}\right)
}

\usepackage{calligra}
\DeclareMathAlphabet{\mathcalligra}{T1}{calligra}{m}{n}
\DeclareFontShape{T1}{calligra}{m}{n}{<->s*[2.2]callig15}{}

\newcommand{\scripty}[1]{\ensuremath{\mathcalligra{#1}}}

\pagestyle{fancy}
\author{Jeremiah Givens}
\newcommand{\subject}{Metric Spaces}
\newcommand{\Date}{9/2/2021} 
\makeatletter
\rhead{{\small\@author}}
\lhead{{\small\subject}}
\chead{{\large Homework 6}}
\cfoot{}
\rfoot{\thepage}
\lfoot{\today}

\renewcommand{\theequation}{\arabic{equation}}

\begin{document}
\section*{Problem 16}
We will start this problem with an easy corollary to the theorem we have about all intervals in $\mathbb{R}$ being connected:
\begin{corollary}
Let $I \subseteq \mathbb{R}$, and let $a, b \in \mathbb{R}$. Then, the spaces $\{(a,  y): y \in I\}$ and $\{(a,  y): y \in I\}$ are connected.
\end{corollary}

\begin{proof}
Define $f:I \to \{(a,  y): y \in I\}$ by $f(y) = (a,  y)$. Clearly $f$ is continuous. Since $I$ is connected, the topological invariance of connectedness tells us that $f(I)$ is connected. Since $f(I) = \{(a,  y): y \in I\}$, we have that $\{(a,  y): y \in I\}$ is connected. The argument for $\{(a,  y): y \in I\}$ is nearly identical.
\end{proof}

\begin{theorem}
For some $z \in \mathbb{R}^2$,  the space $\mathbb{R}^2 \backslash \text{seg}[0, z]$ is connected.
\end{theorem}

\begin{proof}
Let $z \in \mathbb{R}^2$, and let $|z| = d(0, z)$ denote the distance between the origin and $z$. Then, via a rotation about the origin, $\mathbb{R}^2 \backslash \text{seg}[0, z]$ is homeomorphic to $\mathbb{R}^2 \backslash \text{seg}[(0, 0), (|z|, 0)]$. Thus, by the topological invariance of connectedness, it will suffice to show that $\mathbb{R}^2 \backslash \text{seg}[(0, 0), (|z|, 0)]$ is connected.

Let $A_r = \{(x, r): x \in \mathbb{R}\}$ be a horizontal line for $r \in \mathbb{R}\backslash\{0\}$. Let $L = \{(x, 0) \in \mathbb{R}^2: x < 0\}$, and let $R = \{(x, 0) \in \mathbb{R}^2: x > |z| \}$. Define $H = \{A: A = A_r \text{ for some } r\in \mathbb{R}\backslash\{0\} \lor A = L \lor A = R \}$. We have constructed $H$ in such a way that
\begin{align*}
 \bigcup_{A \in H} A = \mathbb{R}^2 \backslash \text{seg}[(0, 0), (|z|, 0)].
\end{align*}
By Corollary 1, we have that $A$ is connected for all $A \in H$. Taking advantage of Corollary 1 once more, we define two connected sets:
\begin{align*}
v_1 &= \{(-1, y) : y \in \mathbb{R}\}\\
v_2 &= \{(|z| + 1, y) : y \in \mathbb{R}\}.
\end{align*}
We have $v_1 \cap A \not = \emptyset$ for all $A \in H \backslash\{R\}$, and we have $v_2 \cap A \not = \emptyset$ for all $A \in H \backslash\{L\}$. Thus, as we proved in class,  $v_1 \cup A$ is connected for all $A \in H \backslash\{R\}$, and $v_2 \cup A$ is connected for all $A \in H \backslash\{L\}$.  Define 
\begin{align*}
F_1 &= \{v_1 \cup A: A \in H \backslash\{R\}\\
F_2 &= \{v_2 \cup A: A \in H \backslash\{L\}.
\end{align*}
By construction,  $A \cap B \not = \emptyset$ for all $A, B \in F_1$, and $A \cap B \not = \emptyset$ for all $A, B \in F_2$. Thus,  for
\begin{align*}
F_3 &= \bigcup_{A \in F_1} A, \text{ and}\\
F_4 &=\bigcup_{A \in F_2} A, 
\end{align*}
we have $F_3, F_4$ are connected. In a similar manner, we have that $F_3 \cap F_4 \not = \emptyset$, since, for example, $A_1 \subseteq F_3$ and $A_1 \subseteq F_4$. Therefore, $F_3 \cup F_4$ is connected.  By design, we have 
\begin{align*}
F_3 \cup F_4 = \mathbb{R}^2 \backslash \text{seg}[(0, 0), (|z|, 0)],
\end{align*}
which leads us to conclude that $\mathbb{R}^2 \backslash \text{seg}[(0, 0), (|z|, 0)]$ is connected.
\end{proof}

\begin{remark}
The intuition behind this proof is breaking $\mathbb{R}^2 \backslash \text{seg}[(0, 0), (|z|, 0)]$ into connected horizontal lines, and then joining them together in such a way to show $\mathbb{R}^2 \backslash \text{seg}[(0, 0), (|z|, 0)]$ is connected. This method is laborious, though fairly straightforward.
\end{remark}

\section*{Problem 17}
\begin{theorem}
Let $X, Y$ be topological spaces and let $f:X \to Y$ be a continuous mapping.  Then $f(C(x))$ is not, in general, a component of $f(x)$ if $C(x)$ is a component of $x$. 
\end{theorem}

\begin{proof}
Let $(a, b) \subseteq \mathbb{R}$ be an interval, and consider the function $f:(a, b) \to \mathbb{R}$, defined by $f(x) = x$. We have that $f$ is continuous (let $\delta = \epsilon$, and continuity follows). Also, $(a, b)$ is a maximal connected subset of $(a, b)$.  Let $x \in (a, b)$. Then,  $(a, b) = C(x)$.  By design, $f((a, b)) = (a, b)$ is a connected set containing $f(x)$. However, $\mathbb{R}$ is a connected set containing $f(x)$,  $f((a, b)) \subseteq \mathbb{R}$, and $f((a, b)) \not = \mathbb{R}$. Therefore, by our theorem on component facts, $f(C(x))$ is not a component of $f(x)$.
\end{proof}

\section*{Problem 18}
\begin{theorem}
$\mathbb{R}$ and $\mathbb{R}^2$ are not homeomorphic.
\end{theorem}

\begin{proof}
Suppose, for the sake of contradiction, that there exists a homeomorphism $f: \mathbb{R} \to \mathbb{R}^2$.  Let $x \in \mathbb{R}$. Since $f^{-1}$ is continuous, $\restr{f^{-1}}{\mathbb{R}^2 \backslash \{f(x)\}}$ is also continuous.  We have $f^{-1}(\mathbb{R}^2 \backslash \{f(x)\}) = \mathbb{R} \backslash \{x\}$. By Theorem 1, $\mathbb{R}^2 \backslash \{f(x)\}$ is connected. Since $\mathbb{R} \backslash \{x\}$ is not an interval, it is not connected. Therefore, we have a continuous function mapping a connected space to a non-connected space, which is a contradiction.  From this, we can conclude that $\mathbb{R}$ and $\mathbb{R}^2$ are not homeomorphic.
\end{proof}

\end{document}