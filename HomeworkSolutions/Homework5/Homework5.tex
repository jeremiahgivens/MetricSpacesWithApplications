\documentclass[10pt,a4paper]{article}
\usepackage[utf8]{inputenc}
\usepackage[a4paper,%
            left=.75in,right=.75in,top=1in,bottom=1in]{geometry}
\setlength{\headsep}{0.25in}

\usepackage{amsthm}

\usepackage{graphicx}
            
\usepackage[english]{babel}

\theoremstyle{theorem}
\newtheorem{theorem}{Theorem}
\newtheorem{lemma}{Lemma}
\newtheorem{corollary}{Corollary}
\newtheorem{case}{Case}

\theoremstyle{definition}
\newtheorem{definition}{Definition}

\usepackage{mathtools}
\DeclarePairedDelimiter\bra{\langle}{\rvert}
\DeclarePairedDelimiter\ket{\lvert}{\rangle}
\DeclarePairedDelimiterX\braket[2]{\langle}{\rangle}{#1 \delimsize\vert #2}

\usepackage{amsmath}
\usepackage{amsfonts}
\usepackage{amssymb}
\usepackage{fancyhdr}

\DeclareMathOperator{\interior}{int}

\newcommand{\Tau}{\mathcal{T}}

\newenvironment{amatrix}[1]{%
  \left(\begin{array}{@{}*{#1}{c}|c@{}}
}{%
  \end{array}\right)
}

\usepackage{calligra}
\DeclareMathAlphabet{\mathcalligra}{T1}{calligra}{m}{n}
\DeclareFontShape{T1}{calligra}{m}{n}{<->s*[2.2]callig15}{}

\newcommand{\scripty}[1]{\ensuremath{\mathcalligra{#1}}}

\pagestyle{fancy}
\author{Jeremiah Givens}
\newcommand{\subject}{Metric Spaces}
\newcommand{\Date}{9/2/2021} 
\makeatletter
\rhead{{\small\@author}}
\lhead{{\small\subject}}
\chead{{\large Homework 5}}
\cfoot{}
\rfoot{\thepage}
\lfoot{\today}

\renewcommand{\theequation}{\arabic{equation}}

\begin{document}
\section*{Problem 13}
\begin{theorem}
The function $f:\mathbb{R} \to \mathbb{R}$ defined by $f(x) = \frac{1}{x + 2}$ is continuous at $x = 1$.
\end{theorem}

\begin{proof}
Let $\epsilon >0$.  Let $\delta_0 >0$.  Then, for all $x \in \mathbb{R}$ with $|x - 1| < \delta_0$, we have
\begin{align*}
|f(x) -f(1)| &= |\frac{1}{x + 2} - \frac{1}{1 + 2}| && \text{This is well defined if we impose } \delta_0 \leq 3 \text{ because } \delta_0 \leq3 \implies x \not = -2\\
&= |\frac{1}{x + 2} - \frac{1}{3}| \\
&= |\frac{3 - x - 2}{3(x + 2)}| \\
&= |\frac{1 - x}{3(x + 2)}| \\
&= \frac{|1 - x|}{|3(x + 2)|} \\
&< \frac{\delta_0}{|3(x + 2)|} \\
&\leq \frac{\delta_0}{3((1 - \delta) + 2)} &&\text{Now imposing further that } \delta_0 \leq 1\\
&= \frac{\delta_0}{9 - 3\delta_0}.
\end{align*}
Now what we want to do is choose a $\delta_0$ such that this expression is less than or equal to $\epsilon$:
\begin{align*}
\frac{\delta_0}{9 - 3\delta_0} &\leq \epsilon\\
\delta_0 &\leq 9 \epsilon - 3 \epsilon \delta_0\\
\delta_0(1 + 3 \epsilon) &\leq 9 \epsilon\\
\delta_0 &\leq \frac{9 \epsilon}{1 + 3 \epsilon}.
\end{align*}
Assigning $\delta = \min\{1, \frac{9 \epsilon}{1 + 3 \epsilon} \}$, we have shown that
\begin{align*}
(\forall x \in \mathbb{R})(|x - 1| < \delta &\implies |f(x) - f(1)| < \epsilon).
\end{align*}
Therefore,  since $\epsilon$ was arbitrary, we have shown that $f$ is continuous at $x = 1$.
\end{proof}

\section*{Problem 14.}
Before we jump into this problem, we are going to prove a useful lemma that allows us to describe the product topology in terms of a simple basis (as apposed to the standard subbasis definition).
\begin{lemma}
Let $\Pi_{k=1}^{n} X_k$ be a product space. Then, the set $\mathcal{B} = \{\Pi_{k=1}^{n} U_k : U_k \in \Tau_k\}$ forms a basis for the product topology. 
\end{lemma}

\begin{proof}
By the usual definition we have
\begin{align*}
\Sigma = \{p_{k}^{-1}(U_k): U_k \in \Tau_k\}
\end{align*}
generates the product topology. Since any basis is automatically a subbasis, it will suffice to show that every element of $\mathcal{B}$ can be written as a union of finite intersections of elements in $\Sigma$(to show that $\mathcal{B}$ generates a coarser topology than $\Sigma$ does),  and that every element of $\Sigma$ is an element of $\mathcal{B}$(to show that $\mathcal{B}$ generates a finer topology than $\Sigma$ does).

Let $p_{k}^{-1}(U_k)$ be an element of $\Sigma$. Written out, this element is 
\begin{align*}
p_{k}^{-1}(U_k) = X_1 \times ...  X_{k-1} \times U_k \times X_{k+1} \times ... \times X_n.
\end{align*}
Thus, since $X_j \in \Tau_j$ for $j \in \{1,..,n\}$, and $U_k \in \Tau_k$, we have that $p_{k}^{-1}(U_k) \in \mathcal{B}$.

Now let $\Pi_{k=1}^{n} U_k \in \mathcal{B}$. Then, we have easily that $\Pi_{k=1}^{n} U_k = \bigcap_{k=1}^{n} p_{k}^{-1}(U_k)$. We have now shown that $\mathcal{B} = \{\Pi_{k=1}^{n} U_k : U_k \in \Tau_k\}$ is a basis for the product topology generated by $\Sigma$.
\end{proof}

\begin{theorem}
Let $X, Y$ be topological spaces and let $f:X \to Y$ be continuous. Show that the mapping $h:X \times Y \to Y \times Y$ defined by $h(x, y) = (f(x), y)$ is continuous. 
\end{theorem}

\begin{proof}
Let $\mathcal{B} = \{U \times V : U, V \in T_Y\}$ be a basis for the product topology on $Y \times Y$. By our continuity facts theorem, it will suffice to show that $h^{-1}(A)$ is open in $X \times X$ for each $A \in B$. Let $U \times V \in \mathcal{B}$. Then
\begin{align*}
h^{-1}(U \times V) &= \{(a, b): a \in X \land b \in V \land f(a) \in U\}\\
&= f^{-1}(U) \times V.
\end{align*}
By continuity of $f$, $f^{-1}(U) \in \Tau_X$. By initial assumption, $V \in \Tau_Y$. Thus, $h^{-1}(U \times V) \in \Tau_X \times \Tau_Y$, and is therefore open.  From this, we can conclude that $h$ is continuous.
\end{proof}

\section*{Problem 15}
\begin{theorem}
Let $\{X_k\}_{k=1}^n$ be a family of topological spaces. Suppose $\Pi_{k=1}^n X_k$ is second countable. Then, $X_k$ is second countable for some $X_k$ for some $k \in \{1,...,n\}$.
\end{theorem}

\begin{proof}
Suppose $\Pi_{k=1}^n X_k$ is second countable.  Then, there exists a countable basis $\mathcal{B} = \{B_n: n \in \mathbb{N}\}$ of $\Pi_{k=1}^n X_k$. We propose that $\{p_{j}(B_n): n \in \mathbb{N}\}$ forms a countable basis of $X_j$ for each $j \in \{1,...,n\}$.

Let $U_j \in \Tau_j$. Then, there exists an indexing set $I \subseteq \mathbb{N}$ such that
\begin{align*}
p_{j}^{-1}(U_j) &= \bigcup_{i\in I} B_i.
\end{align*}
Then, we have
\begin{align*}
U_j &= p_j(\bigcup_{i\in I} B_i)\\
&= \bigcup_{i\in I} p_j(B_i),
\end{align*}
and our proof is complete.
\end{proof}
\end{document}