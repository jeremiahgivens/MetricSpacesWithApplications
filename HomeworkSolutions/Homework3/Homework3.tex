\documentclass[10pt,a4paper]{article}
\usepackage[utf8]{inputenc}
\usepackage[a4paper,%
            left=.75in,right=.75in,top=1in,bottom=1in]{geometry}
\setlength{\headsep}{0.25in}

\usepackage{amsthm}

\usepackage{graphicx}
            
\usepackage[english]{babel}

\theoremstyle{theorem}
\newtheorem{theorem}{Theorem}
\newtheorem{lemma}{Lemma}
\newtheorem{corollary}{Corollary}
\newtheorem{case}{Case}

\theoremstyle{definition}
\newtheorem{definition}{Definition}

\usepackage{mathtools}
\DeclarePairedDelimiter\bra{\langle}{\rvert}
\DeclarePairedDelimiter\ket{\lvert}{\rangle}
\DeclarePairedDelimiterX\braket[2]{\langle}{\rangle}{#1 \delimsize\vert #2}

\usepackage{amsmath}
\usepackage{amsfonts}
\usepackage{amssymb}
\usepackage{fancyhdr}

\DeclareMathOperator{\interior}{int}

\newcommand{\Tau}{\mathcal{T}}

\newenvironment{amatrix}[1]{%
  \left(\begin{array}{@{}*{#1}{c}|c@{}}
}{%
  \end{array}\right)
}

\usepackage{calligra}
\DeclareMathAlphabet{\mathcalligra}{T1}{calligra}{m}{n}
\DeclareFontShape{T1}{calligra}{m}{n}{<->s*[2.2]callig15}{}

\newcommand{\scripty}[1]{\ensuremath{\mathcalligra{#1}}}

\pagestyle{fancy}
\author{Jeremiah Givens}
\newcommand{\subject}{Metric Spaces}
\newcommand{\Date}{9/2/2021} 
\makeatletter
\rhead{{\small\@author}}
\lhead{{\small\subject}}
\chead{{\large Homework 3}}
\cfoot{}
\rfoot{\thepage}
\lfoot{\today}

\renewcommand{\theequation}{\arabic{equation}}

\begin{document}
\begin{definition}
Let $(X, \Tau)$, $(Y, \mathcal{S})$ be topological spaces, and let $f : X \to Y$. We say that $f$ is continuous if and only if 
\begin{equation*}
(\forall G \in \mathcal{S})(f^{-1}(G) \in \Tau).
\end{equation*}
\end{definition}

\begin{definition}[Local Continuity] $f$ is said to be continuous at $x_0 \in X$ if and only if 
\begin{equation*}
(\forall W(f(x_0))(\exists U(x_0))(f(U(x_0)) \subseteq W(f(x_0)).
\end{equation*}
\end{definition}

\section*{Problem 7}
\begin{lemma}
Let $X$ and $Y$ be topological spaces. Let $f:X \to Y$, and let $B \subseteq Y$. Then
\begin{equation*}
f(f^{-1}(B)) \subseteq B.
\end{equation*}
\end{lemma}

\begin{proof}
We have
\begin{align*}
p \in f(f^{-1}(B)) &\iff (\exists x \in f^{-1}(B))(f(x) = p) &&\text{Definition of a set's image}\\
&\iff (\exists x \in X)(f(x) = p \land f(x) \in B)&&\text{Definition of a set's preimage}\\
&\implies p \in B,
\end{align*}
which proves that $f(f^{-1}(B)) \subseteq B$.
\end{proof}

\begin{lemma}
Let $X$ and $Y$ be topological spaces. Let $f:X \to Y$, and let $A \subseteq X$. Then
\begin{equation*}
A \subseteq f^{-1}(f(A))
\end{equation*}
\end{lemma}

\begin{proof}
We have
\begin{align*}
p \in A &\implies (\exists x \in A)(f(p) = f(x)) &&\text{Just let } x = p\\
&\iff f(p) \in f(A)&&\text{Definition of a set's image}\\
&\iff p \in f^{-1}(f(A)),&&\text{Definition of a set's preimage}
\end{align*}
which proves that $A \subseteq f^{-1}(f(A))$.
\end{proof}

\begin{theorem}[Global Continuity Facts]
Let $X$ and $Y$ be topological spaces. Let $f:X \to Y$. Then the following are equivalent:
\begin{enumerate}
\item $f$ is continuous.
\item $f^{-1}(F)$ is closed in $X$ for all $F$ closed in $Y$.
\item $f^{-1}(U)$ is open in $X$ for all $U$ members of a subbasis of $\Tau_Y$.
\item $f(\bar{A}) \subseteq \overline{f(A)}$ for all $A \subseteq X$.
\item $\overline{f^{-1}(B)} \subseteq f^{-1}(\bar{B})$ for all $B \subseteq Y$.
\end{enumerate}
\end{theorem}

\begin{proof}$((4) \implies (5))$.  Suppose $f(\bar{A}) \subseteq \overline{f(A)}$ for all $A \in X$, and let $B \subseteq Y$. We have
\begin{align*}
p \in \overline{f^{-1}(B)} &\implies p \in f^{-1}(f(\overline{f^{-1}(B)})) && \text{By Lemma 2}\\
&\implies p \in f^{-1}(\overline{f(f^{-1}(B))}) && \text{By initial assumption}\\
&\implies p \in f^{-1}(\bar{B}), &&\text{By Lemma 1}
\end{align*}
and we have proven $\overline{f^{-1}(B)} \subseteq f^{-1}(\bar{B})$. Since $B$ was arbitrary, we have proven $(4) \implies (5)$.
\end{proof}

\section*{Problem 8}
\begin{theorem}
Let $(X, d)$ and $(Y, \rho)$ be metric spaces, and let $f : X \to Y$. Then the following are equivalent:
\begin{enumerate}
\item $f$ is continuous at $x_0 \in X$.
\item $(\forall \{x_k\})(x_k \to x_0 \implies f(x_k) \to f(x_0))$.
\end{enumerate}
\end{theorem}

\begin{proof}
$((2) \implies (1))$. We will prove the contrapositive. We have 
\begin{align*}
f \text{ is not continuous at } x_0 \in X &\iff \neg(\forall \epsilon > 0)(\exists \delta >0)(f(B(x_0;\delta)) \subseteq B(f(x_0); \epsilon))\\
&\iff (\exists \epsilon > 0)(\forall \delta >0)(f(B(x_0;\delta)) \not \subseteq B(f(x_0); \epsilon))\\
&\iff (\exists \epsilon > 0)(\forall \delta >0)(\exists x \in B(x_0;\delta))(f(x) \not \in B(f(x_0); \epsilon))\\
&\iff (\exists \epsilon > 0)(\forall \delta >0)(\exists x \in B(x_0;\delta))(\rho(f(x), f(x_0)) \geq \epsilon ).
\end{align*}
Fix this $\epsilon$, and define $\delta_n = 1/n$ for $n \in \mathbb{N}$. Using the Axiom of Choice, we can define a sequence $\{x_k\}_{k=1}^\infty$ where $x_k \in B(x_0; \delta_k)$, and $\rho(f(x_k), f(x_0)) \geq \epsilon$. Thus, we have created a sequence $\{x_k\}_{k=1}^\infty$ that converges to $x_0$, and we have a sequence $\{f(x_k)\}_{k=1}^\infty$ which does not converge to $x_0$. Therefore, we have proven the contrapositive.
\end{proof}

\section*{Problem 9}
\begin{theorem}
Let $(\mathbb{R}, d)$ be a metric space, and let $A \subseteq \mathbb{R}$ be a bounded. Then $\sup A \in \bar{A}$.
\end{theorem}

\begin{proof}
Since $A$ is bounded, the least upper bound property of the real numbers leads us to conclude that $\sup A$ exists. Let $s = \sup A$. Suppose, for sake of contradiction,  that $\sup A \not \in \bar{A}$.  Then, $(\exists U(s) \in \Tau)(U(s) \cap A = \emptyset)$. Since this $U(s)$ is open, we can find a $\delta > 0$ such that $B(s; \delta) \subseteq U(s)$.  This implies that there is no element in $A$ between $s$ and $s - \delta$, which means that $s - \delta$ is an upper bound of $A$. Since $s - \delta$ is an upper bound of $A$, and $s - \delta < s$, $s$ cannot be the supremum of $A$, and we have reached a contradiction. Thus, $\sup A \in \bar{A}$
\end{proof}
\end{document}