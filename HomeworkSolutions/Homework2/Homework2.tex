\documentclass[10pt,a4paper]{article}
\usepackage[utf8]{inputenc}
\usepackage[a4paper,%
            left=1in,right=1in,top=1in,bottom=1in]{geometry}
\setlength{\headsep}{0.25in}

\usepackage{amsthm}

\usepackage{graphicx}
            
\usepackage[english]{babel}

\theoremstyle{theorem}
\newtheorem{theorem}{Theorem}
\newtheorem{lemma}{Lemma}
\newtheorem{corollary}{Corollary}
\newtheorem{case}{Case}

\theoremstyle{definition}
\newtheorem{definition}{Definition}

\usepackage{mathtools}
\DeclarePairedDelimiter\bra{\langle}{\rvert}
\DeclarePairedDelimiter\ket{\lvert}{\rangle}
\DeclarePairedDelimiterX\braket[2]{\langle}{\rangle}{#1 \delimsize\vert #2}

\usepackage{amsmath}
\usepackage{amsfonts}
\usepackage{amssymb}
\usepackage{fancyhdr}

\DeclareMathOperator{\interior}{int}

\newcommand{\Tau}{\mathcal{T}}

\newenvironment{amatrix}[1]{%
  \left(\begin{array}{@{}*{#1}{c}|c@{}}
}{%
  \end{array}\right)
}

\usepackage{calligra}
\DeclareMathAlphabet{\mathcalligra}{T1}{calligra}{m}{n}
\DeclareFontShape{T1}{calligra}{m}{n}{<->s*[2.2]callig15}{}

\newcommand{\scripty}[1]{\ensuremath{\mathcalligra{#1}}}

\pagestyle{fancy}
\author{Jeremiah Givens}
\newcommand{\subject}{Metric Spaces}
\newcommand{\Date}{9/2/2021} 
\makeatletter
\rhead{{\small\@author}}
\lhead{{\small\subject}}
\chead{{\large Homework 2}}
\cfoot{}
\rfoot{\thepage}
\lfoot{\today}

\renewcommand{\theequation}{\arabic{equation}}

\begin{document}
\section*{Problem 4.}
\begin{theorem}
Let $(X,  \Tau)$ be a topological space and let $B \subseteq X$. Then
\begin{equation*}
\bar{B} = \mathbb{C}[\interior(\mathbb{C}B)].
\end{equation*}
\end{theorem}
\begin{proof}
We have 
\begin{align*}
p \in \bar{B} &\iff (\forall U(p) \in \Tau)(U(p) \cap B \not = \emptyset) && \text{We proved this in class.}\\
&\iff (\forall U \in \Tau)(p \in U \implies U \cap B \not = \emptyset)&& \text{Definition of neighborhood}\\
&\iff (\forall U \in \Tau)(U \cap B = \emptyset \implies p \not \in U) && \text{Contrapositive}\\
&\iff (\forall U \in \Tau)(U \cap B = \emptyset \implies p \in \mathbb{C}U)\\
&\iff (\forall U \in \Tau)(U \subseteq \mathbb{C}B \implies p \in \mathbb{C}U)&& U \cap B = \emptyset \iff U \subseteq \mathbb{C}B \\
&\iff (\forall U \in \Tau)(p \in U \implies U \not \subseteq \mathbb{C} B)&& \text{Contrapositive}\\
&\iff p \not \in \interior(\mathbb{C}B) && \text{By definition of interior}\\
&\iff p \in \mathbb{C}[\interior(\mathbb{C}B)] \text{,}
\end{align*}
and we can conclude that $\bar{B} = \mathbb{C}[\interior(\mathbb{C}B)]$.
\end{proof}

\section*{Problem 5.}
\begin{theorem}
Let $(X,  \Tau)$ be a topological space and let $A \subseteq X$. Then
\begin{equation*}
\partial A = \bar{A} \backslash \interior(A)
\end{equation*}
\end{theorem}

\begin{proof}
Proceeding as usual,
\begin{align*}
p \in \partial A &\iff p \in \bar{A} \cap \overline{\mathbb{C}A}\\
&\iff p \in \bar{A} \cap \mathbb{C}[\interior(\mathbb{C}(\mathbb{C}A))] &&\text{Directly applying Theorem 1 from Problem 4}\\
&\iff p \in \bar{A} \cap \mathbb{C}[\interior(A)] && \text{Basic property of complements}\\
&\iff p \in \bar{A} \backslash \interior(A)\text{,} && \text{Definition of set difference}
\end{align*}
and we have shown $\partial A = \bar{A} \backslash \interior(A)$.
\end{proof}

\section*{Problem 6.}
\begin{theorem}
Let $(X,  \Tau)$ be a topological space and let $A \subseteq X$. Then $\interior(A)$ is open.
\end{theorem}

\begin{proof}
By definition of a set's interior, we have $(\forall p \in \interior(A))(\exists U(p) \in \Tau)(U(p) \subseteq A)$.  From this we can define the set 
\begin{align*}
\{U(p) \in \Tau | U(p) \subseteq A \text{ for some } p \in \interior(A)\}.
\end{align*} 
We propose that 
\begin{align*}
\interior(A) = \bigcup \{U(p) \in \Tau | U(p) \subseteq A \text{ for some } p \in \interior(A) \} \text{.}
\end{align*}
We have 
\begin{align*}
p \in \interior(A) &\iff (\exists U(p) \in \Tau)(U(p) \subseteq A)\\
&\iff p \in \bigcup \{U(p) \in \Tau | U(p) \subseteq A \text{ for some } p \in \interior(A) \} \text{.}
\end{align*}
Since any union of sets in $\Tau$ is also in $\Tau$, it follows that $\interior(A)$ is open.
\end{proof}
\end{document}