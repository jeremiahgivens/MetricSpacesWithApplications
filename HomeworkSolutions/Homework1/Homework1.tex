\documentclass[12pt,a4paper]{article}
\usepackage[utf8]{inputenc}
\usepackage[a4paper,%
            left=1in,right=1in,top=1in,bottom=1in]{geometry}
\setlength{\headsep}{0.25in}

\usepackage{amsthm}

\usepackage{graphicx}
            
\usepackage[english]{babel}

\theoremstyle{theorem}
\newtheorem{theorem}{Theorem}
\newtheorem{lemma}{Lemma}
\newtheorem{case}{Case}

\theoremstyle{definition}
\newtheorem{definition}{Definition}

\usepackage{mathtools}
\DeclarePairedDelimiter\bra{\langle}{\rvert}
\DeclarePairedDelimiter\ket{\lvert}{\rangle}
\DeclarePairedDelimiterX\braket[2]{\langle}{\rangle}{#1 \delimsize\vert #2}

\usepackage{amsmath}
\usepackage{amsfonts}
\usepackage{amssymb}
\usepackage{fancyhdr}

\newenvironment{amatrix}[1]{%
  \left(\begin{array}{@{}*{#1}{c}|c@{}}
}{%
  \end{array}\right)
}

\usepackage{calligra}
\DeclareMathAlphabet{\mathcalligra}{T1}{calligra}{m}{n}
\DeclareFontShape{T1}{calligra}{m}{n}{<->s*[2.2]callig15}{}

\newcommand{\scripty}[1]{\ensuremath{\mathcalligra{#1}}}

\pagestyle{fancy}
\author{Jeremiah Givens}
\newcommand{\subject}{Metric Spaces}
\newcommand{\Date}{8/24/2021} 
\makeatletter
\rhead{{\small\@author}}
\lhead{{\small\subject}}
\chead{{\large Homework 1}}
\cfoot{}
\rfoot{\thepage}
\lfoot{\today}

\renewcommand{\theequation}{\arabic{equation}}

\begin{document}
\section*{Problem 1}
\begin{theorem}
Let $f:X \to Y$ be a mapping.  Then
\begin{equation*}
f(f^{-1}(B)) \subseteq B \text{,  for every } B \subseteq Y \text{.}
\end{equation*}
\end{theorem}

\begin{proof}
Let $B \subseteq X$.  By definition of a set's image and a set's preimage,  we have
\begin{align*}
y \in f(f^{-1}(B)) &\iff (\exists x \in f^{-1}(B))(y = f(x))\\
&\iff (\exists x \in X)(y = f(x) \land f(x) \in B)\\
&\implies y \in B \text{.}
\end{align*}
Therefore,  $f(f^{-1}(B)) \subseteq B \text{,  for every } B \subseteq	 Y$.
\end{proof}

\section*{Problem 2} We will start this problem by proving a lemma that we will need for the main theorem.

\begin{lemma}
Let $f:X \to Y$ be a mapping.  Then
\begin{equation*}
f(A \cap B) \subseteq f(A) \cap f(B) \text{,  for all } A,B \subseteq X \text{.}
\end{equation*}
\end{lemma}

\begin{proof}
Let $A,  B \subseteq X$.  By definition of a set's image and the intersection of two sets,  we have
\begin{align*}
y \in f(A \cap B) &\iff (\exists x \in A \cap B)(y = f(x))\\
&\iff (\exists x \in A)(x \in B \land f(x) = y)\\
&\implies (\exists x \in A)(f(x) = y) \land (\exists x \in B)(f(x) = y)\\
&\iff y \in f(A) \land y \in f(B)\\
&\iff y \in f(A) \cap f(B)\text{,}
\end{align*}
and our proof is complete.
\end{proof}

\begin{theorem}
Let $f:X \to Y$ be a mapping.  Then 
\begin{equation}
f \text{is injective if and only if } f(A \cap B) = f(A) \cap f(B) \text{,  for all } A,B \subseteq X.
\end{equation}
\end{theorem}

\begin{proof}
Let $A,B \subseteq X$.  We have already proven that for any function $f: X \to Y$,  $f(A \cap B) \subseteq f(A) \cap f(B)$.  Therefore,  we need to show $f$ is injective if and only if $f(A) \cap f(B) \subseteq f(A \cap B)$. 

Suppose $f$ is injective.  Then
\begin{align*}
y \in f(A) \cap f(B) &\iff y \in f(A) \land y \in f(B) && \\
&\iff (\exists x \in A)(f(x) = y) \land (\exists x \in B)(f(x) = y) &&\\
&\iff (\exists x \in A \cap B)(f(x) = y) && \text{Injectivity of } f \\
&\iff y \in f(A \cap B) \text{.}
\end{align*}
Thus,  $f(A) \cap f(B) \subseteq f(A \cap B)$. 

We will now prove the contrapositive of the reverse direction.  That is,  we will prove that 
\begin{equation} 
\text{f is not injective } \implies  f(A) \cap f(B) \not \subseteq f(A \cap B).
\end{equation}

Suppose $f$ is not injective. Then,  $(\exists x_1,  x_2 \in X)(f(x_1) = f(x_2) \land x_1 \not = x_2)$.  Define $A = \{x_1\}$ and $B = \{x_2\}$.  We have $A,B \subseteq X$,  and $A \cap B = \emptyset$.  By definition of  a sets image,  $f(A \cap B)= f(\emptyset) = \emptyset$. However,  $f(A) \cap f(B) = \{f(x_1) \} \cap \{f(x_2)\} = \{f(x_1) \}$,  since $f(x_1) = f(x_2)$.  Therefore,  $f(A) \cap f(B) \not \subseteq f(A \cap B)$,  and our proof is complete.
\end{proof}

\section*{Problem 3}
We will start this problem with some lemmas that we will use for the proof of the main theorem.

\begin{lemma}
Define $d: \mathbb{R} \to \mathbb{R}$ by $d(x,  y) = |x - y|$.  Then $d$ is a metric for $\mathbb{R}$.
\end{lemma}

\begin{proof}
Let $x,  y \subseteq \mathbb{R}$.  By definition of absolute value,  we have $d(x, y) = x - y \iff x \geq y$,  and $d(x,  y) = y - x \iff y > x$.  By ordering of the reals,  we have $d(x,  y) \geq 0$.  Suppose $x = y$.  Then,  $d(x,  y) = x - y = x - x = 0$.  Now suppose $d(x, y) = 0$.  We can assume,  without loss of generality,  that $x \leq y$.  Then,  from the algebraic properties of $\mathbb{R}$,  
\begin{align*}
d(x,  y) &= 0\\
|x - y| &= 0\\
y - x &= 0\\
y &= x \text{.}
\end{align*}
Therefore,  $d$ satisfies the first requirement of a metric.

Assume,  without loss of generality,  that $x \leq y$.  Then $d(x,  y) = y - x$.  Now,  swap $x$ and $y$ in the above definition of absolute value to see (trivially) that $d(y,  x) = y - x$. Therefore,  $d(x, y) = d(y, x)$,  and $d$ fits the second criteria of a metric.

Now let $x, y,z \in \mathbb{R}$.  Once more,  we can assume that $x \leq z$.  Then,  $d(x, z) = z - x$.  We have three cases to consider:

\item[\bf{Case 1. } $y < x \leq z$.] Then,  
\begin{align*}
d(x,  y) + d(y,  z) &= (x - y) + (z - y)\\
&> (x - y) + (z - y) + 2(y - x)&& \text{because }(y - x) < 0 \\
&= z - x\\
&= d(x,  z) \text{.}
\end{align*}

\item[\bf{Case 2.} $x \leq y \leq z$. ] In a similar fashion,
\begin{align*}
d(x,  y) + d(y,  z) &= (y - x) + (z - y)\\
&= z - x \\
&= d(x,  z) .
\end{align*}

\item[\bf{Case 3.} $x \leq z < y$. ] Once more,
\begin{align*}
d(x,  y) + d(y,  z) &= (y - x) + (y - z)\\
&> (y - x) + (y - z) + 2(z - y) &&\text{because }(z - y) < 0\\
&= z - x\\
&= d(x,  z).
\end{align*}

In all of these cases,  $d(x,  z) \leq d(x,  y) + d(y,  z)$,  which means that $d$ fits the final criteria of a metric.
\end{proof}

\begin{theorem}
Let $X = \{f:[a, b] \to \mathbb{R} | f \text{ is continuous}\}$.  Define
\begin{equation*}
d(f,  g) = \sup \{|f(t) - g(t)| : t \in [a, b] \}.
\end{equation*}
Then $d$ is a metric for $X$.
\end{theorem}

\begin{proof}
To show that $d$ is a metric,  we will show that it fits the three requirements of a metric.  Let $f,  g \in X$.  By Lemma 2,  we have that $|f(t) - g(t)| \geq 0$ for all $t \in [a,  b]$.  Thus,  $d(f,  g) \geq 0$.  Now suppose that $f = g$.  By definition of equality of functions,  we have 
\begin{equation*}
(\forall t \in [a,  b])(f(t) = g(t)).
\end{equation*}
Thus,  by Lemma 2,  we have 
\begin{align*}
\{|f(t) - g(t)| : t \in [a, b] \} &= \{0 \}\\
d(f,  g) &= \sup \{|f(t) - g(t)| : t \in [a, b] \}\\
&= \sup \{0 \} \\
&= 0.
\end{align*}
Now suppose that $\sup \{|f(t) - g(t)| : t \in [a, b] \} = 0$.  Then,  $(\forall t \in [a,  b])(|f(t) - g(t)| \leq 0)$.  By Lemma 2,  we have $(\forall t \in [a,  b])(|f(t) - g(t)| \geq 0)$.  Thus,  $f = g$, and we have shown that $d$ fits the first criteria of a metric.

By Lemma 2,  we have 
\begin{align*}
d(f,  g) &= \sup \{|f(t) - g(t)| : t \in [a, b] \}\\
&= \sup \{|g(t) - f(t)| : t \in [a, b] \}\\
&= d(g,  f) \text{,}
\end{align*}
and $d$ fits the second criteria of a metric.
\end{proof}

Let $f,  g,  h \in X$.  By Lemma 2,  we have
\begin{equation*}
(\forall t \in [a,  b])(|f(t) - g(t)| \leq |f(t) - h(t)| + |g(t) - h(t)|) \text{,}
\end{equation*}
from which it follows that 
\begin{align*}
d(f,  g) &= \sup \{|f(t) - g(t)| : t \in [a, b] \}\\
&\leq \sup \{|f(t) - h(t)| + |g(t) - h(t)|: t \in [a, b] \}.
\end{align*}
Define $s_1 := \sup \{|f(t) - h(t)| + |g(t) - h(t)|: t \in [a, b] \}$,  $s_2 := \sup \{|f(t) - h(t)| : t \in [a, b] \}$,  and $s_3 := \sup \{|g(t) - h(t)| : t \in [a, b] \}$.  By the Extreme Value Theorem,  there exist $t_1,  t_2,  t_3 \in [a,  b]$ such that $|f(t_1) - h(t_1)| + |g(t_1) - h(t_1)| = s_1$,  $|f(t_2) - h(t_2)| = s_2$,  and $|g(t_3) - h(t_3)| = s_3$.  Now suppose,  for sake of contradiction,  that $s_1 > s_2 + s_3$. Then,  
\begin{align*}
s_1 &> s_2 + s_3\\
|f(t_1) - h(t_1)| + |g(t_1) - h(t_1)| &> |f(t_2) - h(t_2)| + |g(t_3) - h(t_3)| \\
|f(t_1) - h(t_1)| + |g(t_3) - h(t_3)| &> |f(t_2) - h(t_2)| + |g(t_3) - h(t_3)| \\
|f(t_1) - h(t_1)|  &> |f(t_2) - h(t_2)| \text{,}
\end{align*}
which contradicts our assumption that $|f(t) - h(t)|$ attains a maximum at $t_2$. Therefore $s_1 \leq s_2 + s_3$,  and $d$ fits the triangle equality
\begin{align*}
d(f,  g) &\leq d(h, g) + d(f, h).
\end{align*}
Finally,  we can conclude that $d$ is a metric for $X$.
\end{document}